%-*- Mode: LaTeX -*-
% For JASIS special OPAC issue '95
%% Desc: JASIS version [replace ``%JASIS'' with ``'' to format for JASIS]

%JASIS\documentstyle[12pt,apa]{article}
\documentstyle[apa]{article}

\input{psfig}
\input{epsf}
%\parindent 0in
\textwidth 6.5truein
%\evensidemargin -.25truein
\oddsidemargin  -.0truein
\topmargin -.8truein
% for JASIS
%JASIS\topmargin -.5truein
%JASIS\renewcommand{\baselinestretch}{2}
\textheight 9.5truein
%\hyphenation{}


\begin{document}

%JASIS\pagestyle{myheadings}
%JASIS\markright{\bf Larson}

\title{Cheshire II: Designing a Next-Generation Online Catalog}

\author{Ray R. Larson \and Jerome McDonough 
	\and Paul O'Leary \and Lucy Kuntz \and Ralph Moon\dag\\
	\\
        School of Information Management and Systems\\
	and The Library\dag\\
        University of California, Berkeley\\
        Berkeley, CA 94720-4600\\
	{\em ray@sherlock.berkeley.edu} }

\date{{\em Revised Version - November 1, 1995}}

{\small \maketitle} \pagenumbering{roman}

%JASIS\newpage
\begin{abstract} 
The Cheshire II online catalog system was designed to provide a bridge
between the realms of purely bibliographical information and the
rapidly expanding full-text and multimedia collections available online.
It is based on a number of national and international standards for
data description, communication and interface technology. The system
uses a client-server architecture with X window client communication
with an SGML-based probabilistic search engine using the Z39.50 
information retrieval protocol.
\end{abstract}

\newpage %\tableofcontents \newpage
\pagenumbering{arabic}

\section{Introduction}

Online public access catalogs have provided access to the collections
of increasing numbers of libraries for over a decade. Indeed, many of
the online catalogs at large research libraries are now over a decade
old, and they have used the same basic search methods, user interface,
and hardware configuration for that entire period.

While there have been various condemnations of online catalogs, or
nostalgic recollections of card catalogs\cite{BAKER} the online
catalog has been embraced by both librarians and library patrons (more
or less happily). The primary effect of library automation, as applied
to the catalog, has been to facilitate rapid and effective access to
the desired items in the collection when the author, title or subject
headings of the item are known to the searcher. However, it has also
been recognized for over a decade that the present generation of
online catalogs in most libraries do not do a very good job of
providing topical or subject access to the collections\cite{MATTHEWS}.

The common result of many subject searches (up to half of such
searches in some systems) is search failure or ``zero results''. The
reasons for this vary from search to search, but they include common
misspelling, lack of knowledge of Boolean logic, and lack of
familiarity with Library of Congress Subject Headings. When the user
does succeed in a subject search, he or she will often be presented
with far too many items to conveniently scan, in an order that has no
relationship to the topical nature of works displayed or the
query. The causes of this information overload vary from system to
system and from search to search, but they include database size (or
the relative collection size for a given topic) and the increasing
numbers of items indexed by a given word in Boolean keyword-based
online catalogs. The overload problem is compounded by the provision
of keyword and heading truncation features in searching, and by
searchers' tendency to use very general wording in their query
formulations.  At the same time, despite the problems they encounter
in subject searching, online catalog searchers use subjects more
frequently than any other access point in the catalog database. Many
studies of online catalog use and users have found that regardless of
the problems involved, subject or topical access is greatly desired
and valued by online catalog users (for more comprehensive review of
the research on subject searching, search failure, and information
overload, see \cite{SCYLLA}).

The online catalog systems in place today are primarily derived from
early information retrieval systems based on Boolean logic and exact
keyword matching. Researchers have suggested that these systems are
deficient for effective subject access for a variety of reasons\cite{HILDRETH}.
These include:

\begin{enumerate}

\item They provide no aid to the searcher in formulating effective
queries. There is usually no attempt to map from the searcher's notion
of a topic to the terms or subject headings actually used to describe
that topic in the database. Likewise, there are usually no facilities
for broadening or narrowing the focus of a search.

\item They do not foster browsing of the database. Most online catalogs 
have no way to exploit the many obvious linkages between database records
(e.g., other items with the same author, class number or subject heading)
without re-typing a complete query. Nor do they provide any simple facility
for requesting ``more like this one''.

\item They do not provide any useful ordering of retrieved records.
Most online catalogs display retrieved records sorted into
author/title order, regardless of the type of search performed. As we
suggest below, a better ordering for topical searches is one based on
the probability of relevance.

\item They do not provide integrated access to information sources. Most
online catalogs are strictly catalogs, and offer access to
bibliographic records exclusively. While {\em digital libraries}
\cite{DIGLIB} have begun providing
a wide variety of network-based information sources including
full-text and multimedia information, the online catalog has not kept
pace with the changing technology.

\end{enumerate}

The Cheshire II project is focussed on the development of a
next-generation online catalog system that addresses these problems
with existing online catalogs. In this paper we will describe the
design of the Cheshire II system and its components. We also
provide a more detailed discussion of some active research 
areas in the design and development of the search engine and 
methods for combining Boolean and probabilistic retrieval techniques.

\section{The Cheshire II System}

The Cheshire II online catalog system was designed to provide a bridge
between existing online catalog technology and databases and the
explosively growing realm of network-based digital libraries with
information resources including full-text and multimedia.  The primary
objectives of the Cheshire II project have been: 1) to develop and
demonstrate a next-generation online catalog system with advanced
searching features using modern workstation and networking technology
in a working library environment, and 2) to evaluate the retrieval
performance and the use and acceptance of this online catalog by
library patrons and remote network users.

Our design has been driven by the belief that many of the failures
of subject access in online catalogs can be alleviated by removing
some of the limitations on the computing technology, information
retrieval methods and user interaction techniques used in earlier
online catalog systems. As pointed out above, most existing online
catalogs are based on previous generations of computing technology,
both hardware and retrieval algorithms, and do not take advantage of
recent advances in computer hardware, software, and networking
technology.

In the Cheshire II system we have incorporated as many of these
advances as possible in order to evaluate their effectiveness in the
context of system performance, and in the use and usability of the
system.  The Cheshire project is intended to help understand and
evaluate a number of practical and research problems in the design and
evaluation of information systems. It provides both a practical
demonstration of the use and effectiveness of an advanced online
catalog system with "state-of-the-art" subject searching capabilities,
and the utility of current standards for information retrieval and
data structuring.

One goal of the system design was to provide an extensible system
that can easily adapt to new types of data, and to provide a
flexible and programmable user interface to display that data.
In order to achieve this goal, we have attempted to incorporate 
appropriate national and international standards into the system 
wherever possible. The Cheshire II system design elements include:

\begin{enumerate}

\item {\em An SGML Database}: SGML\cite{GOLDFARB} is used as the
primary data base format of our underlying {\em search engine}.

\item {\em Z39.50 Client/Server Operation}: The system is based on a 
client/server architecture where the interfaces (clients)
communicate with the search engine (server) using the Z39.50
Information Retrieval Protocol\cite{ANSI}.

\item {\em Boolean and Probabilistic Searching}: The Cheshire II
server (or search engine) supports both conventional Boolean and
probabilistic ``best match'' ranked searching based on estimation of
the probability of relevance for each query/document pair. The server 
permits Boolean and probabilistic elements within the same
query.

\item {\em An X Window-Based GUI}: The user interfaces (Z39.50 clients) 
developed for the Cheshire II system provide a
direct manipulation interface on
X terminals. Support for Mosaic/World Wide Web access is under development.

\item {\em Hypertext linkages and browsing}: 
The Cheshire II search engine and graphical user interface facilitate
browsing through automatically generated hypertext links, and
through ``nearest neighbor'' searches and relevance feedback.

\end{enumerate}

There have been a number of experimental online catalog systems that
have provided ranked retrieval\cite{FOX93,EVAL,OKAPI1,OKAPI2,PORTER}, and
a number of systems provide Z39.50 access, but the Cheshire II system
is the first system to combine all of these design elements.
In the following sections we will discuss each of these design elements
and the active research issues associated with them.

\subsection{Cheshire II and SGML}

\begin{figure}[p]
%\begin{figure}[t]
\begin{center}
\framebox{
\parbox{4.5in}{
\tt{\tiny{
\begin{tabbing}
<USMARC Material="BK" ID="00000007"><leader><LRL>00893</LRL><RecStat>n</RecStat>\\
<RecType>a</RecType><BibLevel>m</BibLevel><UCP></UCP><IndCount>2</IndCount><SFCo\\
unt>2</SFCount><BaseAddr>00289</BaseAddr><EncLevel> </EncLevel><DscCatFm>a</DscC\\
atFm><LinkRec> </LinkRec><EntryMap><FLength>4</Flength><SCharPos>5</SCharPos><ID\\
Length>0</IDLength><EMUCP></EMUCP></EntryMap></Leader><Directry>0010014000000050\\
01700014008004100031010001400072020001500086035002000101035001700121039001800138\\
04000230015605000230017908200170020210000270021924500630024626000400030930000270\\
03494900038003765040051004146500025004658300038004909500034005289500034005629980\\
00700596</Directry><VarFlds><VarCFlds><Fld001>CUBGGLAD1258B</Fld001><Fld005>1994\\
0818092701.0</Fld005><Fld008>840216s1982    nyu      b    00110 eng  </Fld008></\\
VarCFlds><VarDFlds><NumbCode><Fld010 I1="Blank" I2="Blnk"><a>82016816 </a></Fld0\\
10><Fld020 I1="Blank" I2="Blnk"><a>0387907637</a></Fld020><Fld035 I1="Blank" I2=\\
"Blnk"><a>(CU)ocm08762561</a></Fld035><Fld035 I1="Blank" I2="Blnk"><a>(CU)GLAD12\\
58</a></Fld035><Fld039 I1="0" I2="Blnk"><a>2</a><b>3</b><c>3</c><d>3</d><e>3</e>\\
</Fld039><Fld040 I1="Blank" I2="Blnk"><a>DLC</a><c>DLC</c><d>OCL</d><d>CUY</d></\\
Fld040><Fld050 InLofC="Yes" CNSrc="Blnk"><a>QA274.2</a><b>.E44 1982</b></Fld050>\\
<Fld082 Edition="Full" CNSrc="Blnk"><a>519.2/32</a><Two>19</Two></Fld082></NumbC\\
ode><MainEnty><Fld100 NameType="Single" I2=""><a>Elliott, Robert James.</a></Fld\\
100></MainEnty><Titles><Fld245 AddEnty="Yes" NFChars="0"><a>Stochastic calculus\\
and applications /</a><c>Robert J. Elliott.</c></Fld245></Titles><EdImprnt><Fld2\\
60 I1="" I2="Blnk"><a>New York, N.Y. :</a><b>Springer,</b><c>c1982.</c></Fld260>\\
</EdImprnt><PhysDesc><Fld300 I1="Blank" I2="Blnk"><a>viii, 302 p. ;</a><c>25 cm.\\
</c></Fld300></PhysDesc><Series><Fld490 Traced="Differnt" I2="Blnk"><a>Applicati\\
ons of mathematics ;</a><v>18</v></Fld490></Series><Notes><Fld504 I1="Blank" I2=\\
"Blnk"><a>Includes bibliographical references and index.</a></Fld504></Notes><Su\\
bjAccs><Fld650 SubjLvl="NoInfo" SubjSys="LCSH"><a>Stochastic analysis.</a></Fld6\\
50></SubjAccs><AddEnty></AddEnty><LinkEnty></LinkEnty><SAddEnty><Fld830 I1="Blan\\
k" NFChars="0"><a>Applications of mathematics ;</a><v>18</v></Fld830></SAddEnty>\\
<HoldAltG></HoldAltG><Fld9XX><Fld950 I1="Blank" I2="Blnk"><l>ENGI</l><x>220</x><\\
a>QA274.2</a><b>.E44 1982</b></Fld950><Fld950 I1="Blank" I2="Blnk"><l>MATH</l><x\\
>380</x><a>QA274.2</a><b>.E44 1982</b></Fld950></Fld9XX></VarDFlds></VarFlds></U\\
SMARC>\\
\end{tabbing}
}}
}}
\caption{SGML Version of USMARC Record}
\label{sgmlmarc}
\end{center}
\end{figure}

In designing the Cheshire II system we faced the question of how to
provide a search engine that could be used on both simple text and
complex structured records, such as MARC and other bibliographic
records, and also support complex multimedia documents and
databases. After considering the variety of structured and
unstructured data types that we intended to incorporate into the
Cheshire II database, we decided to adopt the Standard Generalized
Markup Language (SGML)\cite{GOLDFARB} as the fundamental data storage
type for Cheshire II.  All of the data in the Cheshire II database are
stored as tagged, SGML documents (see Figure \ref{sgmlmarc} for an example
of a MARC record in SGML format).

The adoption of SGML has provided a number of benefits for the
Cheshire II system. The primary benefit has been that through the use
of SGML tagging for all data in the database, and the adoption of the
SGML Data Type Definition (DTD) language to define the structure of
each data file, we have a common format for data types ranging from
full-text documents, structured bibliographic records, to complex
hypertext and multimedia documents (using the HTML DTD that defines
the elements of World Wide Web (WWW) ``pages'').  This has important
economies in the development process and in the addition of new
types of data to the system. 

Virtually all data manipulation for the database has been generalized
as processes acting on SGML tags or sets of tags. Instead of having to
develop new routines to manipulate each sub-element of a new data
type, the developer only needs to provide a DTD and a conversion
routine to convert the new data type to SGML. The built-in file
manipulation and indexing routines can then extract and index any
tagged sub-elements of the datatype for access. For example, after a
MARC record such as the one shown in Figure \ref{sgmlmarc} has been
tagged, creating an index on a new element (such as keywords extracted
from sub-tag ``$<$a$>$'' within tag ``$<$FLD830$>$'') simply involves
specifying that within a configuration file and running the extraction
and indexing processes. Similarly, data extraction and indexing can be
performed for any other tags specified in any SGML DTD, such as
extracting the footnotes in a full-text document, or the captions of
pictures in a WWW page.

SGML is also used as the basic format of Cheshire II configuration
files. These files define the physical database elements of the
Cheshire II system, including the locations of data files, which SGML
DTD describes the file, and information on which indexes to create and
the elements they should contain.  Figure \ref{sgml.subjindex} shows
a portion of a configuration file defining a MARC database, and the
index definitions for subject indexing in that file. Each database
may contain multiple named files, and each of those may have any 
number of indexes extracted from them. The Cheshire II search engine
relies on the configuration file to define all of the accessible
elements of the database. Adding or deleting elements is as simple
as changing the configuration file. Thus, it would be easy to change
Cheshire II system an advanced catalog accessing a MARC database to
system to a personal information retrieval system for a researcher's 
field notes, by making some changes to the basic configuration file.

\begin{figure}[p]
%\begin{figure}[t]
\begin{center}
\framebox{
\parbox{4.5in}{
\tt{\tiny{
\begin{tabbing}
<!-- This is a sample configuration file for Cheshire II -->\\
<DBCONFIG>\\
<!-- The first filedef -->\\
<FILEDEF TYPE=SGML>\\
<!-- filetag is the "shorthand" name of the file -->\\
<FILETAG> bibfile </FILETAG>\\
<!-- filename is the full path name of the file -->\\
<FILENAME> /usr3/cheshire2/indexing/TESTDATA/morerecs.sgml</FILENAME>\\
<!-- fileDTD is the full path name of the file's DTD -->\\
<FILEDTD> /usr3/cheshire2/new/sgml/USMARC07.DTD </FILEDTD>\\
<!-- assocfil is the full path name of the file's Associator -->\\
<ASSOCFIL> /usr3/cheshire2/indexing/TESTDATA/morerecs.sgml.asso</ASSOCFIL>\\
<!-- history is the full path name of the file's history file -->\\
<HISTORY> /usr3/cheshire2/indexing/TESTDATA/morerecs.sgml.history</HISTORY>\\
<!-- The following are the index definitions for the file -->\\
<INDEXES>\\
...\\
<!-- Subject index definition -->\\
<INDEXDEF ACCESS=BTREE EXTRACT=KEYWORD NORMAL=STEM>\\
<INDXNAME> /usr3/cheshire2/indexing/TESTDATA/dictionary.subject </INDXNAME>\\
<INDXTAG> subject </INDXTAG>\\
<!-- The following INDXMAP items provide a mapping from the SUBJECT tag to -->\\
<!-- the appropriate Z39.50 BIB1 attribute numbers      -->\\
<INDXMAP><USE> 21 </USE><POSIT> 3 </posit> <struct> 6 </struct> </INDXMAP>\\
<INDXMAP><USE> 26 </USE><POSIT> 3 </posit> <struct> 6 </struct> </INDXMAP>\\
<INDXMAP><USE> 25 </USE><POSIT> 3 </posit> <struct> 6 </struct> </INDXMAP>\\
<INDXMAP><USE> 27 </USE><POSIT> 3 </posit> <struct> 6 </struct> </INDXMAP>\\
<INDXMAP><USE> 28 </USE><POSIT> 3 </posit> <struct> 6 </struct> </INDXMAP>\\
\\
<!-- The associator file for the index linking the termid with postings -->\\
<INDASSOC> /usr3/cheshire2/indexing/TESTDATA/mainfile.subj.idxasso </INDASSOC>\\
<!-- The postings file for the index containing all term/document/freq info -->\\
<INDXPOST> /usr3/cheshire2/indexing/TESTDATA/mainfile.subj.idxpost </INDXPOST>\\
<!-- The stoplist for this file -->\\
<STOPLIST> /usr3/cheshire2/indexing/TESTDATA/titlestoplist </STOPLIST>\\
<!-- The INDXKEY area contains the specifications of tags in the doc -->\\
<!-- that are to be extracted and indexed for this index    -->\\
<INDXKEY>\\
<TAGSPEC>\\
<!-- Here we used wildcards '.' to indicate all tags starting "FLD6"     -->\\
<!-- followed by any two characters in the main file should be extracted -->\\
<!-- for this index -->\\
<FTAG>FLD6.. </FTAG> \\
</TAGSPEC> </INDXKEY> </INDEXDEF>\\
...\\
\end{tabbing}
}}}}
\caption{Subject Index from Configuration File}
\label{sgml.subjindex}
\end{center}
\end{figure}


\subsection{The Cheshire II Search Engine}

The original Cheshire catalog system was designed several years ago to
test the use of probabilistic information retrieval methods upon MARC
bibliographic data. In studies that compared these probabilistic
retrieval algorithms to Boolean and other IR methods such as the
vector space model, it was found that a combination of {\em
classification clustering} and the probabilistic algorithms provided
the best retrieval performance for a test database of MARC
data\cite{CLASSCLUS,EVAL}. In classification clustering, all of the
title and subject words for each record in a given class number are
used to provide access points to that topical area.

For the Cheshire II project the search engine was redesigned to
support a variety of search and browsing capabilities. We have
included facilities for both probabilistic and Boolean searching in
Cheshire II. This was driven by the realization that there are
different types of search tasks that are best handled by different
retrieval methods.  Therefore, we provide support for such methods as
authority-controlled name searching and other conventional online
catalog search features, such as ``exact title'' and ``exact subject''
matching capability and the ability to store and retrieve both Boolean
and probabilistic ``result sets'' and use them in subsequent queries.

The search engine also supports various methods for translating a
searcher's query into the terms used in indexing the database. These
methods include elimination of unused words using field-specific
stopword lists, particular field-specific query-to-key conversion or
``normalization'' functions, algorithms for reducing significant words
to their {\em roots} or {\em stems} by converting suffix variations,
such as plural forms of a word, to a single form, as well as support
for mapping database and query text words to single forms based on the
WordNet dictionary and thesaurus.

However, the primary functionality that distinguishes the Cheshire II
search engine from conventional {\em second generation} Boolean online
catalog systems is the support for probabilistic searching on any
indexed element of the database. This means that a natural language
query can be used to retrieve the {\em best} matching records (or
clusters, for clustered access points) in the database, and not just
the exact Boolean matches. In both cluster searching and direct
probabilistic searching of the database, the Cheshire II search engine
supports {\em relevance feedback} so that any items found in an
initial search (Boolean or probabilistic) can be selected and used as
queries in a relevance feedback search. 

This is an extension of the two-stage search method developed in the
Cheshire prototype. In the prototype probabilistic retrieval methods
were used to match the searcher's query with a set of {\em
classification clusters}, the searcher then selected the clusters that
appeared relevant and they were combined with the initial query and
used to re-rank the database, so that records were retrieved in
decreasing order of probable relevance to the searcher's initial query
statement combined with the broad classes selected in the first stage.
This two-stage search method appeared to assist the searcher in
subject focusing and topic/treatment discrimination\cite{CLASSCLUS}. The
cluster search method is still available in Cheshire II, but is now augmented
by direct probabilistic searching of the database.

\subsubsection{Probabilistic Retrieval in Cheshire II}

The probabilistic retrieval algorithm used in the Cheshire II search
engine is based on the {\em staged logistical regression} algorithms
developed by Berkeley researchers and shown to provide excellent
full-text retrieval performance in the TREC evaluation of full-text IR
systems\cite{SLR,TREC2,TREC3}.  Formally, the probability of relevance
given a particular query and a particular record in the database 
$P(R \mid Q, D)$ is calculated and the records are presented to the user ranked
in order of decreasing values of that probability.  In the Cheshire II
system $P(R \mid Q, D)$ is calculated as the ``log odds'' of relevance
$\log O(R \mid Q, D)$, where for any events $A$ and $B$ the odds $O(A
\mid B)$ is a simple transformation of the probabilities $\frac{P(A
\mid B)}{P(\overline{A} \mid B)}$. The Staged Logistic Regression
method provides estimates for a set of coefficients, $c_{i}$, associated
with a set of $S$ statistics, $X_{i}$, derived from the query and database,
such that

\begin{equation}
\log O(R \mid Q,D) \approx c_{0} \sum_{i=1}^{S} c_{i} X_{i}
\label{equ1}
\end{equation}

where $c_{0}$ is the intercept term of the regression.

For the set of $M$ {\em terms} (i.e., words, stems or phrases) that
occur in both a particular query and a given document, the equation
used in estimating the probability of relevance for the Cheshire II
search engine is essentially the same as that used in \cite{TREC2}
where the coefficients were estimated using relevance judgements from
the TIPSTER test collection:

\begin{description}
\item [$X_{1} = \frac{1}{M}\sum_{j=1}^{M} log QAF_{t_{j}}$]. This 
is the log of the absolute frequency of occurrence for term $t_{j}$ 
in the query averaged over the $M$ terms in common between the query and the
document. The coefficient $c_1$ used in the current version of the
Cheshire II system is 1.269.
\item [$X_{2} = \sqrt{QL}$]. This is square root of the query length
(i.e., the number of terms in the query disregarding stopwords). The
$c_2$ coefficient used is -0.310.
\item [$X_{3} = \frac{1}{M}\sum_{j=1}^{M} log DAF_{t_{j}}$]. This is 
is the log of the absolute frequency of occurrence for term $t_{j}$ 
in the document averaged over the $M$ common terms. The 
$c_3$ coefficient used is 0.679.
\item [$X_{4} = \sqrt{DL}$]. This is square root of the document length.
In Cheshire II the raw size of the document in bytes is used for the
document length. The $c_4$ coefficient used is -0.0674.
\item [$X_{5} = \frac{1}{M}\sum_{j=1}^{M} log IDF_{t_{j}}$]. This is 
is the log of the {\em inverse document frequency}(IDF) for term $t_{j}$ 
in the document averaged over the $M$ common terms. 
IDF is calculated as the total number of documents in the database,
divided by the number of documents that contain term $t_{j}$
The $c_5$ coefficient used is 0.223.
\item [$X_{6} = log M$]. This is the log of the number of common terms.
The $c_6$ coefficient used in Cheshire II is 2.01.

\end{description}

The Cheshire II search engine calculates
all matching functions at the point of retrieval, rather than 
pre-computing portions of the functions. Only the fundamental statistics
(such as raw term frequency) are stored in the database, making it easy
to apply a different algorithm to the same database without re-indexing.

Probabilistic searching, as noted above, requires only a natural language
statement of the searcher's topic, and thus no formal query language
or Boolean logic is needed for such searches. However, the Cheshire II
search engine also supports complete Boolean operations on indexed
elements in the database. One active area of research is examining
the combination of Boolean and probabilistic ranked elements within
the same query, which we discuss in the following section.

\subsubsection{Combining Boolean and Probabilistic Searching}

The Cheshire II system provides users with the ability to search using
either natural language queries with probabilistic ranking of search
results or conventional Boolean queries and term matching, as well as
the option to use both types of searches simultaneously.  Although
these are implemented within a single process, they comprise two
parallel {\em logical} search engines.  Each logical search engine
produces a set of retrieved documents.  When a user chooses only one
type of search strategy then the result set of that search is
presented directly to the user, either a probabilistically ranked set
or an unranked Boolean result set. When the user queries the database
using the parallel search strategies the two result sets are merged
and presented to the user as a single set.

Each of these two types of queries can be thought of as distinct
representations of the user's abstract information need -- each with 
advantages for particular types of searches. The parallel querying process
allows the user to state the information need in more than one form, thus
giving the whole system a more complete statement of that need.
This also allows users to take advantage of the strengths of each
search strategy and to create queries tailored to their particular
requirements. For example, in searching for a known item or known author,
explicit Boolean query formulations are effective.
Alternately, in subject searching, when users rarely know the indexing
or classification terms used to describe the desired but unknown items,
probabilistic matching of queries to documents is more effective. 

From the user's perspective however, there may not be a clear
distinction between these types of searches.  Users of ranked
retrieval systems (usually those experiences with Boolean systems)
have often expressed a desire to refine a ranked retrieval by using
the more restrictive Boolean operators in conjunction with the ranking
mechanism.  This combination of retrieval methods would allow the
user, for example, to disambiguate the sense of a keyword in the
Boolean query with a description of its intended sense or context in
the probabilistic query. In effect, the system would raise the
relevance ranking of documents associated with the desired sense of
the exact match keyword.  Another example of the usefulness of this
parallel search strategy is the case where the documents retrieved
under the heading of an especially prolific corporate author were
ranked according to a probabilistically defined topic statement. In
each case the user is able to specify a particular information need
more precisely, and to retrieve a better ranking of relevant
documents, than either one of the two types of queries and search
strategies would afford.

Besides allowing the user greater flexibility, the motivation for
using two search methods follows from the observation that no single
retrieval algorithm has been consistently proven to be better than any
other algorithm for all types of searches. By combining the retrieved
sets from these two search strategies, we hope to leverage the
strengths and reduce the limitations of each type of retrieval system.
In general, the more evidence the system has about the relationship
between a query and a document, the more accurate it will be in
predicting the probability that the document will satisfy the user's
need\cite{BELKIN}.  Other researchers\cite{KEEN} have shown that
additional information about the location and proximity of Boolean
search terms can be used to provide a ranking score for a set of
documents.  Recent IR models have shown that the exact match Boolean
retrieval status can be used as evidence of the probability of
relevance in the context of a larger network of probabilistic
evidence\cite{FUHR,TURTLE90,TURTLE92}. In the same way, we treat the
set of documents resulting from the exact match Boolean query as a
special case of a probabilistically ranked set, with each retrieved
document having an equal rank. The Boolean result set is combined with
the ranked result set from the probabilistic query to form a single
ranked result set using evidence from both logical retrieval engines
to determine a more accurate probability of relevance.

\subsubsection{Merging the Results of Two Logical Search Engines}

The primary problem in this parallel retrieval strategy
is in determining the relationship between the results of the two
retrieval systems. This relationship can be seen as the dependency
between two types of evidence in a probabilistic inference network
\cite{FUHR,TURTLE90}. The dependency cannot be formally specified
before an actual query is submitted because the system cannot predict
the features of the particular query.  For example, if the user enters
the same term in each side of the parallel query then the result sets
could be expected to have a very high degree of statistical
dependence: the presence of a document in the exact match result set
predicts a higher rank for that document in the probabilistic result
set. Alternately, if the user submits terms from widely divergent
subject areas on each side of a parallel query, then the result sets
would be expected to have some lesser dependency relationship; the
retrieval of a document by exact match to the one term predicts little
or nothing about that document's ranked relevance to the other term.

Another theoretical issue involves the difference in the underlying
models of exact match and probabilistic retrieval. In its basic form,
the Boolean exact match model is only concerned with the simple
matching of query terms to document or index terms; the model says
nothing about the user's subjective relevance judgments. Probabilistic
models, on the other hand, include a model of the user's judgments by
attempting to predict relevance with statistical evidence.
Complicating things further, there are two general interpretations of
the ordering of probabilistic ranked
sets\cite{BOOKSTEIN85,FOXKOLL}. One treats the document rankings as
representing degrees of relevance. The weight or rank assigned to any
document in the retrieved set reflects a relative measure of the
document's relevance. In the other interpretation, the rank of each
document represents a probability that a document is relevant in
absolute terms. If we focus on the user's perspective however, this
distinction is not critical, because users of IR systems tend to
interpret probability rankings as relative relevance rankings, or
measures of potential usefulness in satisfying the need expressed in
the query\cite{BOOKSTEIN83}. Extending this idea, from the user's
point of view we can consider membership in the Boolean retrieved set
as a prediction of a user's relevance judgment. This is in keeping
with Turtle and Croft's incorporation of the Boolean retrieval model
into the probabilistic inference network
model\cite{TURTLE90,TURTLE92}.  We will merge the result sets of our
Boolean and probabilistic search engines with the justification that
Boolean retrieval is a special case in the probabilistic model.

In order to merge the two sets into one ranked set we need to 
anticipate some relationship between the two sets. As discussed above, we 
cannot predict the dependency relationship between the terms in each
part of the query. But we can specify a relationship between the Boolean
and probabilistic results based on a simple analysis of 
the Boolean portion of the query. This analysis classifies the Boolean
query, and by extension the Boolean retrieved set, based on the
relative advantages and disadvantages of an exact match search strategy 
compared to a probabilistic search strategy. We then assign relative value 
to the Boolean retrieved set, in the form of a coefficient in the merging 
algorithm. 

The objective is to weight the result sets based on the type of
search, favoring each system or retrieval strategy where it is most
effective. Known item searches, such as author or exact title
searches, are given the highest weighting in the merging process.  In
this case the Boolean retrieved set is given a greater weight than the
probabilistic retrieved set.  The result sets from title or abstract
keyword searches are assigned less of a value than those from known
item searches.  As a starting point for testing, results from these
queries are given equal value with the results of probabilistic
queries.  Finally, result sets from keyword searches in the full text
of a document -- the weakest type of search in a Boolean system -- are
assigned weights of less value than the result set of a probabilistic
query. In this case, the merged result set weighted in favor of the
more useful probabilistic search engine, and augmented somewhat by the
occurrence of a keyword in the text of the document.  We expect this
method of merging the results of Boolean and probabilistic queries to
be especially useful in improving the results of this type of keyword
Boolean retrieval strategy. These merging strategies, and the
coefficients for merging different types of searches, are being
evaluated for future implementation if testing shows a significant
improvement over the current strategy. At present, combined
probabilistic and Boolean search results are evaluated using the
assumption that the Boolean retrieved set has an estimated $P(R \mid
Q_{bool},D) = 1.0$ for each document in the set, and $0$ for the rest
of the collection. The final estimate for the probability of relevance
used for ranking the results of a search combining Boolean and
probabilistic strategies is simply:

\begin{equation}
P(R \mid Q,D) = P(R \mid Q_{bool},D) P(R \mid Q_{prob},D) \nonumber
\end{equation}

where $P(R \mid Q_{prob},D)$ is the probability estimate from the
probabilistic portion of the search, and $P(R \mid Q_{bool},D)$ the
estimate from the Boolean. This has the effect of restricting the
results to those items that match the Boolean portion, with ordering
based on the probabilistic portion.

\subsubsection{Browsing and Relevance Feedback}

One desirable property of an online catalog is to provide methods of
open-ended, exploratory browsing through the database. This feature is
being implemented in the Cheshire II search engine is several ways.

One obvious way to provide browsing is to permit the user to follow 
static or dynamically established linkages between records in the database, 
(e.g., jump to the next record with the same subject heading) in order 
to find items with some association to those previously retrieved. In
Cheshire II this functionality is being provided by a combination of
selection mechanisms in the user interface and exact Boolean search methods
in the search engine (this hypertext mechanism is described further
in the following section). 

A more interesting method for browsing is the inclusion of
relevance feedback in the Cheshire II search engine. In the current
implementation, relevance feedback is implemented as probabilistic
retrieval based on extraction of content-bearing elements (such as
titles, subject headings, etc.) from any items that have already been
seen and selected by a user. Thus, any citation or document seen by
the user can become the basis for a {\em nearest neighbor} search, where
it is used as a query to find those records in the database most
similar in content to the one specified. Similarly, multiple records
may be selected and submitted for feedback searching. In this case
the contents of all those records are merged into a single query and
submitted for searching. In the current implementation, generating a
feedback search is accomplished by parsing the selected record(s) and
extracting the record elements specified for the index used for topical
searching (as specified in the database configuration file). Each of these
record elements is combined to form a single query, which is then
submitted to the same probabilistic retrieval process described above.
At the present time we do not use any methods
for eliminating poor search terms from the selected records, nor
special enhancements for terms common between multiple selected 
records\cite{FEEDBACK}, but we plan to experiment further with various
enhancements to our relevance feedback method.


\subsection{The Cheshire II Client Interface}

\begin{figure}[t]
%\begin{figure}[p]
\begin{center}
\fbox{
%\epsfxsize=4.5in
%\epsfysize=4in
%\epsfbox{sgC2rank.eps}
Place figure near here
}
\caption{Cheshire II Client Performing Mixed Probabilistic/Boolean
Search}
\label{ranked}
\end{center}
\end{figure}

\begin{figure}[t]
%\begin{figure}[p]
\begin{center}
\fbox{
%\epsfxsize=4.5in
%\epsfysize=4in
%\epsfbox{sgC2bool.eps}
Place figure near here
}
\caption{Cheshire II Client Performing Pure Boolean Search}
\label{Boolean}
\end{center}
\end{figure}

The evolution of the Cheshire II client interface has been driven by a
tension between two desires on the part of the designers.  The first
of these desires was to produce a client interface that was more than
simply a GUI for traditional OPAC searching; we hoped to produce a
client which would support end-user searching with a variety of Z39.50
servers, any of which might support many different search engines and
produce several different document formats.  Our second desire was, to
the extent possible, to minimize the cognitive load on users wishing
to search this diverse set of resources by providing a single,
coherent user interface for interacting with all of them.  Our hope
was to produce a client capable of searching either an OPAC system or
an image database (and displaying results from those searches) with
equal facility and with minimal reconfiguration of the interface
itself.

There were several other design criteria that we formulated for the
client interface.  While we hoped to limit reconfiguration of the
interface as the user moved from server to server, we also wanted to
ensure that screen space was not wasted in presenting mechanisms for
search interaction that were irrelevant in the context of a particular
client-server session.  As an example, if a user switched from a
search session with the Cheshire II server to one with the University
of California Melvyl Z39.50 system, those aspects of the interface
necessary for specifying probabilistic queries would no longer be
useful and should be removed from the display. Obviously, our hopes in
this regard were to some degree in direct conflict with our desire to
minimize changes to the interface when moving from server to server,
and negotiating between these goals has proved one of the more
difficult aspects of the client design.  In addition to the goals
already stated, we also hoped to:

\begin{enumerate}

\item minimize use of additional windows during users' interactions
with the client in order to allow them to concentrate on formulating
queries and evaluating the results, and not expend additional mental
effort and time switching their focus of attention from the search
interface to display clients;

\item provide functions not immediately related to searching,
such as print and e-mail facilities, to facilitate users' ability to
'take the results home'; and

\item design a help system within the interface that would assist
users not only in the mechanics of operating the Cheshire II client,
but also in the more general tasks of selecting appropriate resources
for searching, formulating appropriate queries, and employing various
search tactics.

\end{enumerate}

In particular, by monitoring users' search results and, when possible,
providing context-sensitive suggestions on how to improve a query, we
planned to provide an interface that would assist users in both
refining a search over time and extracting useful information as their
search progressed, as suggested by Bates\cite{BATES}.

To date, we believe we have been reasonably successful in negotiating
among these goals.  Fig. \ref{ranked} and fig. \ref{Boolean} show the
reconfigurations to the interface which occur in switching from a
search with the Cheshire II server (where a mixed probabilistic and
Boolean search is being performed) to one with Melvyl Z39.50 server,
which only supports Boolean queries.\footnotemark\footnotetext{The
pull-down menus have been left on-screen to show how the change in
available Boolean search indexes is conveyed to the user.} The text
entry area and ranking type selection button for specifying
probabilistic queries is removed, and in its place two additional
Boolean index specification/text entry areas are provided.  The
mechanism for selecting a Boolean index (a pull-down menu to the left
of the applicable text entry area) is the same in both instances,
although the list of indexes is altered to reflect the indexes
available with the current server.  The mechanism for specifying
Boolean operators (AND/OR buttons between the text entry areas) is
also the same, although two additional buttons are provided in the
pure Boolean interface to enable more complex Boolean queries.  

This reconfiguration of index names and searching features will use
the Z39.50 v.3 ``Explain'' database to discover the characteristics of
the server and adapt the interface to it.  The Explain database is a
special database with record formats and searchable elements defined
in the Z39.50 v.3 standard\cite{ANSI,LYNCH}. It contains information about
the server, its databases, and search elements available in those
databases, in a standard machine-readable form. For older servers with
no Explain database, manually constructed tables of information about
the server are used in Cheshire II for known databases.

One additional interaction feature to be noted in the figures is the
dynamically generated hypertext links associated with each name and
subject heading in the displayed records (indicated by raised and 
highlighted text). Each of these is a button that submits a
Boolean query consisting of the highlighted text with the appropriate
index specification. This dynamic hypertext mechanism is based on the
client's ability to identify these elements in both SGML records from
the Cheshire II server, and in MARC records from other Z39.50 servers.
The client software can then treat each of the texts as a button and
associate an action (submitting a new query) with each one. This permits
very simple browsing of the database by following subject heading or
author links. 

\begin{figure}[t]
%\begin{figure}[p]
\begin{center}
\fbox{
%\epsfxsize=4.5in
%\epsfysize=4in
%\epsfbox{sgC2mail.eps}
Place figure near here
}
\caption{Cheshire II Client E-Mail Facility}
\label{email}
\end{center}
\end{figure}

Additional functionality beyond searching and browsing has been
relatively easy to implement.  Functions for printing, e-mailing and
saving records are all available when records are displayed, and the
user has the option of acting on either the entirety of the current
record display or a subset thereof by selecting individual records
using the "select" buttons on each record (visible in
Fig. \ref{ranked} next to the record numbers).  Figure \ref{email}
shows the client's e-mail facility, which includes the ability for the
user to provide additional text in forwarding selected records to a
particular e-mail account.

The Cheshire II client interface has been primarily implemented using
the interpreted Tcl/Tk language\cite{TCL}, with a variety of
lower-level functions, including the majority of the Z39.50 client
interactions, written in the C programming language.  This combination
has proven quite successful in both providing the ability to rapidly
prototype and modify the graphic user interface to accommodate new
features (such as the result summarization and reporting found in the
OASIS system\cite{BUCKLAND}, and maintain a relatively high level of
performance for the Z39.50 client-server interactions.  The
combination of Tcl/Tk and the workstation hardware being used for the
evaluation experiments permits the use of multimedia information
sources including graphics and sound and will permit display of
mathematical formulae and non-roman characters.  We are also
considering altering the existing help facilities, which use Tcl/Tk
text-tagging features for enhanced graphic display and hypertext
links, to support display of SGML documents.

In addition to the Cheshire II client interface, complete access to
the Cheshire II server is available through other Z39.50 clients. The
Cheshire II server also provides support for the HTTP protocol via an
HTTP-to-Z39.50 gateway, giving access to popular WWW clients like
Mosaic and Netscape. This interface (using HTML forms for data entry
elements) provides remote network users many of the same search
features as the full client described above, with some loss of
integration and ease of interactivity. Because HTTP is a stateless
protocol, with each query/response pair considered a complete
transaction, the ability to do relevance feedback is very limited in
the current WWW implementation.


\section{Evaluation Objectives}

One of the primary goals of the Cheshire II project has been to
produce a system that can be used in an actual library setting, and to
evaluate the user's behavior with, and responses to the system
(particularly with regard to its advanced retrieval methods).  In
evaluating a system like Cheshire II, there are several different
aspects to consider.  First, there is the performance of the system
itself.  This includes both efficiency and effectiveness.  Next, there
is the user interface and how well it functions.  Finally, there is
the user and determining user satisfaction and search patterns.

Each category breaks down further into specific evaluation goals.  The
efficiency of the system will be measured in terms of its response
time.  That is, how long it takes between the time a query is entered
and the time results are displayed.  Evaluation of system
effectiveness will be based on calculations of precision and recall
using standard IR test collections and also by using selected queries
from users and expert evaluation of search results.  In addition,
overall user satisfaction will be considered.  Another potential
measure of system effectiveness will be calculated using the
proportion of records in a result set which are saved or sent to the
user through electronic mail.  This is a crude, but potentially
helpful way to estimate the usefulness of the records to the user.

The issues surrounding evaluation of the interface include the ease
with which users learn how to use the system and how well the users
can accomplish their tasks.  The help system and how easy it is for
users to correct their mistakes will also be evaluated. The system
will also be available to network users via WWW browsers such as
Mosaic or Netscape, which do not provide the primary graphical
interface discussed above. This will allow us to evaluate the
interface features presented by by means of comparing the experiences
of both types of users.

The search patterns of the users are of great interest.  Demographic
information such as age, gender, and academic area will be collected
in order to explore possible differences in searching styles, success,
and satisfaction.  In addition, the overall relative use of the
different search capabilities will be determined.  That is, the amount
of Boolean searching will be compared to probabilistic searching and
the use of searching clusters will be compared to direct ranking.
Note that this information will come from direct observation, via the
transaction logs, and will not depend on the user knowing what type of
search is being done. The usefulness of the various indices is also of
interest.  Frequency of use, search results, and user satisfaction in
this context will all be examined.

Two primary methods for evaluation will be used.  The first involves
transaction monitoring and logging of significant events in the users'
interaction with the system.  These transaction logs are recorded
automatically by the system (at both the server or search engine, and
in the client for local users).  The second method is an online
questionnaire presented for users to complete at the end of a search
session.  With questionnaire administration handled entirely online,
network users at remote locations can participate in the evaluation of
the system and its use. This would be much more difficult to
accomplish with a paper questionnaire.

The main drawback to these evaluation methods is that there is no
direct contact with the user.  Thus, there is no way to gain insight
into the thought processes of the user or any other background
information not specifically requested in the questionnaire. We plan
to remedy this lack with interviews of a subset of local users to
supplement the questionnaire and transaction data.

\section{Conclusions}

The design and development of the Cheshire II system has concentrated
on constructing a system that incorporates a variety of components
into a synergistic whole. The development of each of the system components
described above has involved a taking a model, standard, or prototype
element and then extending and adapting that element to conform to an
overall structure composing our vision of the next generation of
online catalogs and similar online information systems. In this
process we have found many benefits, as well as numerous difficulties,
from basing portions of the technology on standards such as Z39.50 and SGML. 

The principle benefits of adopting standard-based technology have been: 

\begin{enumerate}
\item The availability of precise and exacting specifications for 
elements of the technology. For SGML and Z39.50, in particular, the
standards present the appropriate behavior of conforming software in
great detail. 

\item The availability of supporting applications and tools for 
working with standards-based information. SGML, for example, has a 
number of public domain and commercial tools available including
validating parsers, editors and sgml document presentation tools.
 
\item The ability to interoperate with other systems that conform to
all or part of the same standards. As an example of this, we were 
able to begin development of our user interface while the search engine
was still in development by using Z39.50 to interact with other 
search engines over the internet.

\item Standards-based technology can be more easily shared with others,
and those working within a standardized framework benefit from a 
wider community of users and developers working on similar problems.

\end{enumerate}

There are also some drawbacks to using standards-based technologies in
the design and development process. 

\begin{enumerate}

\item Standards like SGML and Z39.50 are complex, and developing
a system that conforms to these standards is a much more time consuming
and difficult task than it would be to develop a non-conforming and
non-standard system.

\item Standards are evolving over time, and thus offer something of a 
``moving target'' for developers. This often raises the issue of whether
the system should conform to a previous version of a standard, or 
attempt to support the incompletely defined ``next version''.

\item Not all desirable features of an information retrieval system 
like Cheshire II are supported in the current version of standards like
Z39.50. In particular, full support for ranked retrieval and relevance feedback
are not in the current Z39.50 standard and were added as non-standard 
extensions to Z39.50 support in the Cheshire II client and server. Thus,
to be an interesting research system as well as a standards compliant
system, while not antithetical goals, are often competing goals.

\item There are many different standards (and de facto standards) in existence
and designers are often forced to choose among them. There is some danger
of making a ``wrong choice'' and being left with a system that is completely
compliant with a standard that nobody uses.

\end{enumerate}

In general, we believe that the benefits of using standards-based
technology outweigh the problems. The decision to develop
standards-based technology in the Cheshire II project has been a good
approach to system specification, design and development. The
standards provide a solid base of functionally that is extended and
enhanced by the addition of research-driven extensions and
enhancements.

In the introduction we described a number of deficiencies with existing
online catalog systems in most libraries. For each of those deficiencies,
the Cheshire II system has provided a remedy:

\begin{enumerate}

\item The Cheshire II system attempts to aid to the searcher in 
formulating effective
queries. This is done by using stemming and probabilistic ``best match''
algorithms, and by the use of classification clusters to help focus
topical searches.

\item The system fosters browsing of the database. We provide relevance
feedback and ``nearest neighbor'' searching for any record displayed to
the user. We also provide ``point and click'' hypertext searching in
in user interface to retrieve items with the same authors or subjects as
those selected.

\item The system provides an ordering of retrieved records in topical
searches based on the estimated probability of relevance.

\item The system provides support for a wide variety of data types
stored as tagged SGML records. This provides a general search and
retrieval engine that can be used for complete digital library 
systems\cite{DIGLIB} including
full-text and multimedia information, and not just the online catalog.

\end{enumerate}

As of this writing, the Cheshire II project is actively researching
many of the issues described in this paper. We are combining work on
database structures and algorithms for probabilistic information
retrieval, advances and extensions to standard information retrieval
protocols, graphical user interfaces, and user evaluation in a single
project.  The Cheshire II system is being installed in the UC Berkeley
Mathematics, Statistics and Astronomy library and will soon be available to
library users. We plan to publish further descriptions of our findings
on retrieval algorithms, use of SGML structured documents as database
objects, user interfaces, and user reactions to the Cheshire II
advanced online catalog.

\section{Acknowledgements}

The work described in this paper was sponsored by a College Library
Technology and Cooperation Grants Program, HEA-IIA, Research and
Demonstration Grant (\#R197D30040) from the U.S. Department of
Education.  The authors would also like to thank the anonymous
reviewers for helpful suggestions on improving this paper.


\begin{thebibliography}{99}

\bibitem[\protect\citeauthoryear{ANSI/NISO}{1995}]{ANSI}
ANSI/NISO Z39.50-1995. (1995).
{\em Information Retrieval (Z39.50): Application Service Definition and
Protocol Specification (ANSI/NISO Z39.50-1995)}. Bethesda, MD: NISO
Press. (available from ftp://ftp.loc.gov/pub/z3950/official)

\bibitem[\protect\citeauthoryear{Baker}{1994}]{BAKER}
Baker, N. (1994).
\newblock DisCards.
\newblock {\em New Yorker}, 70(7), 64-86.

\bibitem[\protect\citeauthoryear{Bates}{1989}]{BATES}
Bates, M.~J. (1989).
\newblock The design of browsing and berrypicking techniques for
the online search interface.
\newblock {\em Online Review}, 13, 407-424.

\bibitem[\protect\citeauthoryear{Belkin {\em et~al.}}{1994}]{BELKIN}
Belkin, N.J., Kantor, P., Cool, C., \& Quatrain, R. (1994). 
\newblock Combining Evidence for Information Retrieval.
\newblock In: D.~K. Harman (Ed.) {\em Second Text Retrieval Conference (TREC-2), Gaithersburg, MD, 
USA, 31 Aug.-2 Sept. 1993)} (pp. 35-44).  Washington : NIST.


\bibitem[\protect\citeauthoryear{Bookstein}{1983}]{BOOKSTEIN83}
Bookstein, A. (1983). 
\newblock Outline of a general probabilistic retrieval
model. 
\newblock {\em Journal of Documentation}, 39, 63-72. 

\bibitem[\protect\citeauthoryear{Bookstein}{1985}]{BOOKSTEIN85}
Bookstein, A. (1985). 
\newblock Probability and fuzzy-set applications to 
information retrieval. 
\newblock {\em Annual Review of Information Science and Technology},
     20, 117-151. 

\bibitem[\protect\citeauthoryear{Buckland {\em et al.}}{1993}]{BUCKLAND}
Buckland, M.~K., Butler, M.~H., Norgard, B.~A., \& Plaunt, C.~P. (1993).
\newblock OASIS: Prototyping graphical interfaces to networked information.
\newblock In: {\em Integrating Technologies, Converging Professions  
(Proceedings of the 56th ASIS Annual Meeting, Columbus, Ohio, Oct. 24-28,
1993)} (pp. 204-210). Medford, NJ : Learned Information.

\bibitem[\protect\citeauthoryear{Cooper {\em et~al.}}{1992}]{SLR}
Cooper, W.~S., Gey, F.~C., \& Dabney, D.~P. (1992).
\newblock Probabilistic Retrieval Based on Staged Logistic Regression.
\newblock In: {\em SIGIR '92 (Proceedings of the Fifteenth Annual International
ACM SIGIR Conference on Research and Development in Information Retrieval,
Copenhagen, Denmark, June 21-24, 1992)} (pp. 198-210). New York: ACM.

\bibitem[\protect\citeauthoryear{Cooper {\em et~al.}}{1994a}]{TREC2}
Cooper, W.~S., Gey, F.~C. \& Chen, A. (1994a).
\newblock Full Text Retrieval based on a Probabilistic Equation with
Coefficients fitted by Logistic Regression.
\newblock In: D.~K. Harman (Ed.) {\em Second Text Retrieval Conference (TREC-2), Gaithersburg, MD, 
USA, 31 Aug.-2 Sept. 1993}, NIST-SP 500-215, (pp. 57-66).  Washington : NIST.

\bibitem[\protect\citeauthoryear{Cooper {\em et~al.}}{1994b}]{TREC3}
Cooper, W.~S., Chen, A. \& Gey, F.~C. (1994b).
\newblock Experiments in the Probabilistic Retrieval of Full Text Documents
\newblock In: {\em Text Retrieval Conference (TREC-3) Draft Conference
Papers,} Gaithersburg, MD : National Institute of Standards and Technology.

\bibitem[\protect\citeauthoryear{Fox {\em et~al.}}{1993}]{FOX93}
Fox, E.~A., France, R.~K., Sahle, E., Daoud, A. \& Cline, B.~E. (1993).
\newblock Development of a Modern OPAC: From REVTOLC to MARIAN.
\newblock IN: {\em SIGIR '93 (Proceedings of the Sixteenth Annual International
ACM SIGIR Conference on Research and Development in Information Retrieval,
Pittsburgh, June 27-July 1, 1993)} (pp. 248-259). New York: ACM.

\bibitem[\protect\citeauthoryear{Fox {\em et~al.}}{1995}]{DIGLIB}
Fox, E.~A., Akscyn, R.~M., Furuta, R.~K. \& Leggett, J.~J. (Eds) (1995).
\newblock Digital Libraries (special issue)
\newblock {\em Communications of the ACM}, 38(4), 23-96.

\bibitem[\protect\citeauthoryear{Fox \& Koll}{1988}]{FOXKOLL}
Fox, E.~A. \& Koll, M.~B. (1988). 
\newblock Practical enhanced Boolean retrieval: 
experiences with the SMART and SIRE systems. 
\newblock {\em Information Processing \&
Management}, 24(3), p. 257-67. 

\bibitem[\protect\citeauthoryear{Fox \& Shaw}{1994}]{FOXSHAW}
Fox, E.~A. \& Shaw, J.~A. (1994). 
\newblock Combination of Multiple Searches.
\newblock In: D.~K. Harman (Ed.) {\em Second Text Retrieval Conference (TREC-2), Gaithersburg, MD, 
USA, 31 Aug.-2 Sept. 1993)} (pp. 243-252).  Washington : NIST.

\bibitem[\protect\citeauthoryear{Fuhr}{1992}]{FUHR} 
Fuhr, Norbert (1992).
\newblock Probabilistic Models in Information Retrieval.
\newblock {\em Computer Journal}, 35, 243-55.

\bibitem[\protect\citeauthoryear{Goldfarb}{1990}]{GOLDFARB}
Goldfarb, C.~F. (1990).
\newblock {\em The SGML handbook.}
\newblock New York : Oxford University Press.

\bibitem[\protect\citeauthoryear{Hildreth}{1989}]{HILDRETH}
Hildreth, C.~R. (1989). 
\newblock OPAC research: laying the groundwork for
future OPAC design.
\newblock In C.~R. Hildreth (Ed.), {\em The Online
Catalogue: Development and Directions} (pp. 1-24). London: The
Library Association.

\bibitem[\protect\citeauthoryear{Keen}{1992}]{KEEN}
Keen, E.~M. (1992).
\newblock Term Position Ranking: Some new test results.
\newblock In: {\em SIGIR '92 (Proceedings of the Fifteenth Annual International
ACM SIGIR Conference on Research and Development in Information Retrieval,
Copenhagen, Denmark, June 21-24, 1992)} (pp. 66-76). New York: ACM.

\bibitem[\protect\citeauthoryear{Larson}{1991a}]{SCYLLA} 
Larson, R.~R. (1991a).
\newblock Between Scylla and Charybdis: Subject searching in the online
catalog.
\newblock {\em Advances in Librarianship}, 15, 175-236.

\bibitem[\protect\citeauthoryear{Larson}{1991b}]{DECLINE} 
Larson, R.~R. (1991b).
\newblock The decline of subject searching: Long-term trends and patterns
of index use in an online catalog.
\newblock {\em Journal of the American Society for Information Science}, 
42, 197-215.

\bibitem[\protect\citeauthoryear{Larson}{1991c}]{CLASSCLUS} 
Larson, R.~R. (1991c).
\newblock Classification Clustering, Probabilistic Information Retrieval,
and the Online Catalog.
\newblock {\em Library Quarterly}, 61, 133-173.

\bibitem[\protect\citeauthoryear{Larson}{1992}]{EVAL} 
Larson, R.~R. (1992).
\newblock Evaluation of Advanced Retrieval Techniques in an Experimental 
Online Catalog.
\newblock {\em Journal of the American Society for Information Science}, 43,
34-53.

\bibitem[\protect\citeauthoryear{Lynch}{1995}]{LYNCH}
Lynch, D. (1995).
\newblock {\em Implementing Explain}. available from 
ftp://ftp.loc.gov/pub/z3950/articles/denis.ps

\bibitem[\protect\citeauthoryear{Matthews}{1983}]{MATTHEWS}
Matthews, J.~R., Lawrence, G.~S., \& Ferguson, D.~K. (1983).
\newblock {\em Using Online Catalogs: A Nationwide Survey.}
\newblock New York: Neal-Schuman Publishers.

\bibitem[\protect\citeauthoryear{Ousterhout}{1994}]{TCL}
Ousterhout, J.~K. (1994).
\newblock {\em Tcl and the Tk Toolkit}
\newblock Reading, Mass. : Addison-Wesley.

\bibitem[\protect\citeauthoryear{Porter}{1988}]{PORTER}
Porter, M. \& Galpin, V. (1988).
\newblock Relevance feedback in a public access catalogue
for a research library: Muscat at the Scott Polar Research Institute.
\newblock {\em Program}, 22, 1-20.

\bibitem[\protect\citeauthoryear{Salton \& Buckley}{1990}]{FEEDBACK}
Salton, G. \& Buckley, C. (1990).
\newblock Improving Retrieval Performance by Relevance Feedback.
\newblock {\em Journal of the American Society for Information Science},
41, 288-297.

\bibitem[\protect\citeauthoryear{Turtle \& Croft}{1990}]{TURTLE90}
Turtle, H.~R.\& Croft, W.~B. (1990).
\newblock Inference Networks For Document Retrieval.
\newblock In: J. Vidick (Ed.){\em Proceedings of the 13th International Conference on Research
and Development in Information Retrieval. (Proceedings of the 13th International
Conference on Research and Development in Information Retrieval, Brussels,
Belgium, 5-7 Sept. 1990)} (pp. 1-24). New York: ACM.

\bibitem[\protect\citeauthoryear{Turtle \& Croft}{1992}]{TURTLE92} 
Turtle, H.~R.\& Croft, W.~B. (1992).
\newblock A Comparison of Text Retrieval Models.
\newblock {\em Computer Journal}, 35, p. 279-90.

\bibitem[\protect\citeauthoryear{Walker}{1987}]{OKAPI1}
Walker, S. (1987).
\newblock OKAPI: Evaluating and enhancing an experimental online catalog.
\newblock {\em Library Trends}, 35, 631-645.

\bibitem[\protect\citeauthoryear{Walker}{1989}]{OKAPI2}
Walker, S. (1989).
\newblock The Okapi online catalogue research projects.
\newblock In: C.~R. Hildreth (Ed.),
{\em The Online Catalogue: Development and Directions}
(pp. 84-106).London: The Library Association.

\end{thebibliography}

\end{document}

